\section{Introduction to Rendering}
Il \textbf{Rendering} è il processo che converte i dati in input in immagini.
Paradigmi del Rendering:
\begin{itemize}
    \item Ray Tracing
    \item Path Tracing
    \item Rasterization-based Pipeline
\end{itemize}
per visualizzare scene 3D abbiamo bisogno di trasformale in immagini sintetiche che possono essere mostrate sullo schermo.
Esistono diversi algoritmi per trasformare una scena 3D in una raster image, in seguito veranno presentati due algoritmi appartenenti a due famiglie differenti:
\begin{itemize}
    \item Ray tracing algorithms
    \item Rasterization-based algorithms
\end{itemize} 
\subsection{Pinhole Camera}
La metafora usata per descrivere la relazione tra l'osservatore e la scena è quella della camera virtuale